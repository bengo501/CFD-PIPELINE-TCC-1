\chapter{Materiais e Métodos}
\label{cap:materiais}

% ============================================================
% VISÃO GERAL
% ============================================================
\section{Visão Geral do Sistema}

O sistema desenvolvido neste trabalho consiste em um pipeline automatizado para simulações CFD de leitos empacotados, composto por múltiplos módulos integrados. A arquitetura geral está ilustrada na Figura \ref{fig:arquitetura_geral}.

% TODO: Adicionar figura da arquitetura
% \begin{figure}[htb]
%     \centering
%     \includegraphics[width=0.9\textwidth]{figuras/arquitetura_geral.pdf}
%     \caption{Arquitetura geral do sistema}
%     \label{fig:arquitetura_geral}
% \end{figure}

\section{Metodologia de Desenvolvimento}

\subsection{Abordagem Ágil}

O desenvolvimento seguiu princípios ágeis com iterações de 1-2 semanas (sprints). Utilizou-se uma adaptação da metodologia Scrumban, combinando elementos do Scrum (sprints planejados) e Kanban (fluxo contínuo).

\subsection{Ferramentas de Desenvolvimento}

\begin{itemize}
    \item \textbf{Controle de Versão}: Git e GitHub
    \item \textbf{IDE}: VSCode com extensões para Python, JavaScript e LaTeX
    \item \textbf{Gerenciamento de Tarefas}: GitHub Projects (Kanban)
    \item \textbf{Documentação}: Markdown e LaTeX
\end{itemize}

% ============================================================
% ARQUITETURA DO SISTEMA
% ============================================================
\section{Arquitetura do Sistema}

\subsection{Arquitetura em Camadas}

O sistema foi projetado seguindo uma arquitetura em camadas, separando responsabilidades:

\begin{enumerate}
    \item \textbf{Camada de Apresentação}: Interface web (React)
    \item \textbf{Camada de API}: Backend RESTful (FastAPI)
    \item \textbf{Camada de Negócios}: Serviços de processamento
    \item \textbf{Camada de Dados}: PostgreSQL + MinIO
    \item \textbf{Camada de Processamento}: Workers assíncronos (Celery)
\end{enumerate}

\subsection{Tecnologias Selecionadas}

A Tabela \ref{tab:tecnologias} apresenta as tecnologias selecionadas e suas justificativas.

\begin{table}[htb]
\centering
\caption{Tecnologias utilizadas no projeto}
\label{tab:tecnologias}
\begin{tabular}{lll}
\toprule
\textbf{Componente} & \textbf{Tecnologia} & \textbf{Versão} \\
\midrule
Linguagem Backend & Python & 3.11 \\
Framework API & FastAPI & 0.104 \\
Banco Relacional & PostgreSQL & 16 \\
ORM & SQLAlchemy & 2.0 \\
Cache/Fila & Redis & 7.2 \\
Task Queue & Celery & 5.3 \\
Object Storage & MinIO & Latest \\
Frontend Framework & React & 18.2 \\
Build Tool & Vite & 5.0 \\
Visualização 3D & Three.js & r158 \\
Parser Generator & ANTLR & 4.13.1 \\
Modelagem 3D & Blender & 4.2 \\
CFD Solver & OpenFOAM & 11 \\
Containerização & Docker & 24.0 \\
Orquestração & Docker Compose & 2.23 \\
\bottomrule
\end{tabular}
\end{table}

% ============================================================
% PIPELINE DE PROCESSAMENTO
% ============================================================
\section{Pipeline de Processamento}

O pipeline completo é ilustrado na Figura \ref{fig:pipeline_completo}.

\subsection{Etapas do Pipeline}

\begin{enumerate}
    \item \textbf{Especificação de Parâmetros}: Usuário define parâmetros via DSL ou interface web
    \item \textbf{Compilação DSL}: Parser ANTLR valida e gera JSON normalizado
    \item \textbf{Geração de Geometria}: Blender cria modelo 3D com simulação física
    \item \textbf{Preparação CFD}: OpenFOAM gera malha e configura caso
    \item \textbf{Simulação}: Solver executa cálculo numérico
    \item \textbf{Pós-Processamento}: Extração de métricas e validação
    \item \textbf{Armazenamento}: Persistência de dados e arquivos
    \item \textbf{Visualização}: Apresentação de resultados na interface web
\end{enumerate}

% ============================================================
% VALIDAÇÃO
% ============================================================
\section{Metodologia de Validação}

\subsection{Comparação com Equação de Ergun}

Os resultados de perda de carga obtidos nas simulações CFD são comparados com valores previstos pela equação de Ergun (Eq. \ref{eq:ergun}). O desvio relativo é calculado como:

\begin{equation}
\text{Desvio} = \frac{|\Delta P_{CFD} - \Delta P_{Ergun}|}{\Delta P_{Ergun}} \times 100\%
\label{eq:desvio}
\end{equation}

Desvios inferiores a 10\% são considerados aceitáveis, conforme literatura \cite{dixon2006}.

\subsection{Estudo de Independência de Malha}

Para garantir que os resultados não dependem do refinamento da malha, foi realizado um estudo de independência utilizando o método GCI (\textit{Grid Convergence Index}) \cite{roache1994}.

% ============================================================
% CASOS DE TESTE
% ============================================================
\section{Casos de Teste}

\subsection{Configurações Avaliadas}

Foram definidos três casos de teste representativos:

\begin{enumerate}
    \item \textbf{Caso 1}: Leito cilíndrico, $D = 50$ mm, $H = 100$ mm, $d_p = 5$ mm, 500 partículas
    \item \textbf{Caso 2}: Leito cilíndrico, $D = 100$ mm, $H = 200$ mm, $d_p = 10$ mm, 800 partículas
    \item \textbf{Caso 3}: Leito cilíndrico, $D = 75$ mm, $H = 150$ mm, $d_p = 7.5$ mm, 650 partículas
\end{enumerate}

\subsection{Propriedades do Fluido}

Em todos os casos, utilizou-se água como fluido de trabalho:
\begin{itemize}
    \item Densidade: $\rho = 1000$ kg/m³
    \item Viscosidade dinâmica: $\mu = 0.001$ Pa·s
\end{itemize}

\subsection{Condições de Contorno}

\begin{itemize}
    \item \textbf{Entrada}: Velocidade uniforme (0.01, 0.05, 0.1 m/s)
    \item \textbf{Saída}: Pressão atmosférica (0 Pa relativo)
    \item \textbf{Paredes}: Condição de não-deslizamento
\end{itemize}

% ============================================================
% MÉTRICAS DE AVALIAÇÃO
% ============================================================
\section{Métricas de Avaliação}

\subsection{Performance do Sistema}

\begin{itemize}
    \item Tempo de geração de geometria
    \item Tempo de geração de malha
    \item Tempo de simulação CFD
    \item Tempo total de pipeline
\end{itemize}

\subsection{Qualidade dos Resultados}

\begin{itemize}
    \item Desvio em relação à equação de Ergun
    \item Qualidade da malha (skewness, aspect ratio)
    \item Convergência da solução numérica
\end{itemize}

\subsection{Usabilidade}

\begin{itemize}
    \item Facilidade de especificação de parâmetros
    \item Clareza da visualização de resultados
    \item Tempo de resposta da interface web
\end{itemize}

