\chapter{Introdução}
\label{cap:introducao}

Os leitos empacotados são parte do dia a dia da engenharia química e aparecem em estudos relacionados a absorção, destilação, reações heterogêneas e sobre transferência de calor e massa. Representá-los bem faz diferença direta no desempenho e resultado dos processos científicos e industriais. Dessa forma, a computação gráfica e as simulações ganharam protagonismo nessa área. Entre as abordagens que estão disponíveis para serem utilizadas, temos a Computational Fluid Dynamics (CFD), que nos permite enxergar, de forma detalhada os campos de velocidade e pressão em geometrias que antes dependiam quase só de correlações empíricas.

Apesar de todo o potencial que esse modelo nos traz, preparar estudos utilizando CFD para leitos empacotados ainda consome tempo e carece de padronização. Criar geometrias realistas, gerar malhas adequadas e encadear todas as etapas até a análise final costuma depender de scripts rodados de forma local, conhecimento técnico em diversos softwares e muitas operações manuais. No meu laboratório de Iniciação Científica, vejo as dificuldades vivenciadas pelo meu orientador e seu doutorando em relação ao empacotamento geométrico, isso acaba por desmotivar e dificultar os estudos e as comparações consistentes entre simulações.

Diante desse cenário, estou propondo um pipeline automatizado e reprodutível que integra desde a descrição do problema até a visualização dos resultados. O fluxo começa em uma linguagem de domínio específico (DSL), na qual descrevo partículas, arranjos, fluido, regime e objetivos. A partir dela, o sistema irá traduzir as especificações para geometrias físicas no geradas no Blender, executar os casos/simulações no OpenFOAM, armazenar metadados e artefatos em um banco de dados e disponibilizar um dashboard interativo com visualização 3D dos modelos e a possibilidade de visualizar gráficos de análises e comparações. Toda a solução será rodada de forma conteinerizada e orquestrada por Docker Compose, o que melhora portabilidade e repetibilidade entre diferentes máquinas.

Essa nova abordagem que eu proponho, combate esses gargalos recorrentes, de forma que o pipeline padroniza as entradas, reduz o trabalho manual, preserva histórico de versões e parâmetros, facilita auditoria e comparação entre cenários. O uso de ferramentas de código aberto, acaba por favorecer a colaboração da ciência aberta, enquanto a conteinerização do pipeline, resulta na mitigação do custo de instalação de programas e amplia a acessibilidade da experiência. Para validar o meu sistema, utilizo referências, como a correlação de Ergun, e realizo estudos de independência de malha com o índice de Convergência de Malha (GCI), usado para verificar a precisão dos resultados de simulações numéricas, como as realizadas em dinâmica dos fluidos computacional, dessa forma, assegurando consistência física.

Para guiar o texto, organizo este TCC como uma narrativa do desenvolvimento desse projeto. Primeiro apresento o problema, as minhas motivações e os objetivos. Depois reúno o referencial teórico sobre escoamento em meios empacotados, fundamentos de CFD e práticas de verificação e validação. Em seguida eu descrevo a arquitetura e a implementação do pipeline, detalhando a DSL, geração geométrica utilizando scripts Python, malha, automações e visualização. Por fim, faço uma discussão sobre a validação da metodologia, trago estudos de caso, resultados, limitações encontradas e caminhos futuros, com foco em ampliar casos de uso e robustez do sistema.

% ============================================================
% DEFINIÇÃO DO PROBLEMA
% ============================================================
\section{Definição do Problema}

Pesquisadores de engenharia química que estudam leitos empacotados costumam enfrentar quatro obstáculos práticos:

\begin{enumerate}
    \item \textbf{Geração de geometria fisicamente coerente}: parede/estrutura cilíndrica, tampas e partículas feita manualmente, consumindo tempo em leitos pequenos ou grandes \cite{blender2021}.
    
    \item \textbf{Preparação da simulação CFD}: muitos arquivos e parâmetros (malha, contornos, esquemas numéricos), suscetível a erros e difícil de reproduzir sem histórico claro \cite{openfoam2023}.
    
    \item \textbf{Organização de resultados}: variáveis calculadas, logs, malhas e campos pouco padronizados, o que dificulta comparações e recuperação de histórico.
    
    \item \textbf{Comunicação dos achados}: dispersa (pastas locais, prints, planilhas), sem integração de dados e visualização consolidada \cite{fastapi2021}.
\end{enumerate}

% ============================================================
% OBJETIVOS
% ============================================================
\section{Objetivos}

\subsection{Objetivo Geral}

Desenvolver um pipeline conteinerizado e reprodutível para simulações CFD de leitos empacotados, integrando geração de geometria, simulação numérica, armazenamento de dados e visualização de resultados em uma plataforma web unificada, utilizando exclusivamente softwares de código aberto.

\subsection{Objetivos Específicos}

\begin{enumerate}
    \item Desenvolver uma Domain Specific Language (DSL) para descrição declarativa de parâmetros de leitos empacotados, com validação sintática e semântica
    
    \item Implementar geração automática de geometrias 3D utilizando Blender em modo headless com simulação física de empacotamento por corpo rígido
    
    \item Integrar o OpenFOAM para automatização de simulações CFD, desde a geração de malha (blockMesh e snappyHexMesh) até a solução numérica (simpleFoam)
    
    \item Projetar e implementar um banco de dados relacional (PostgreSQL) para armazenamento de metadados, variáveis calculadas e rastreabilidade de execuções
    
    \item Desenvolver um sistema de armazenamento escalável para arquivos grandes (MinIO compatível com S3) organizando artefatos por usuário e execução
    
    \item Criar uma API RESTful (FastAPI) com autenticação JWT para gerenciamento do pipeline e geração de URLs temporárias para arquivos
    
    \item Implementar um sistema de filas assíncronas para processamento de jobs em background (Redis)
    
    \item Desenvolver uma interface web interativa (React + Three.js + Plotly) para visualização 3D de geometrias e análise gráfica de resultados
    
    \item Containerizar toda a aplicação utilizando Docker Compose para garantir reprodutibilidade e portabilidade entre ambientes
    
    \item Realizar validação dos resultados através da comparação com a equação de Ergun e estudos de independência de malha (GCI)
\end{enumerate}

% ============================================================
% JUSTIFICATIVA
% ============================================================
\section{Justificativa}

Para validar o meu sistema, utilizo referências, como a correlação de Ergun \cite{ergun1952}, e realizo estudos de independência de malha com o índice de Convergência de Malha (GCI) \cite{roache1994}, usado para verificar a precisão dos resultados de simulações numéricas, como as realizadas em dinâmica dos fluidos computacional, dessa forma, assegurando consistência física.

Este trabalho se justifica pela crescente necessidade de ferramentas que aumentem a produtividade em pesquisa e desenvolvimento de processos industriais. A automação proposta oferece os seguintes benefícios:

\begin{itemize}
    \item \textbf{Reprodutibilidade}: Todos os parâmetros e configurações são documentados automaticamente com versionamento através de hash do params.json canônico, seed do empacotamento e versões dos containers
    
    \item \textbf{Escalabilidade}: Permite execução de dezenas ou centenas de simulações parametrizadas através de sistema de filas
    
    \item \textbf{Rastreabilidade}: Histórico completo de execuções, variantes e resultados em banco de dados relacional
    
    \item \textbf{Colaboração}: Múltiplos usuários podem compartilhar configurações e resultados através da API autenticada
    
    \item \textbf{Validação}: Comparação automática com correlações empíricas estabelecidas (Ergun) e índices de qualidade de malha
    
    \item \textbf{Acessibilidade}: Interface web torna a ferramenta acessível sem instalação local complexa, reduzindo barreiras de entrada
    
    \item \textbf{Código Aberto}: Uso exclusivo de ferramentas open source favorece a ciência aberta e colaboração científica
\end{itemize}

% ============================================================
% ESCOPO
% ============================================================
\section{Escopo}

O escopo deste trabalho foca em:

\begin{itemize}
    \item \textbf{Geometrias}: Leitos cilíndricos preenchidos com partículas (esferas, cubos, cilindros e planos) definidas via DSL; o modelo 3D inclui tampas inferior e superior
    
    \item \textbf{Regime de Escoamento}: Incompressível, laminar ou turbulento (RANS)
    
    \item \textbf{Solver CFD}: simpleFoam do OpenFOAM para regime permanente
    
    \item \textbf{Empacotamento}: Física de corpo rígido no Blender para arranjo realista de partículas
    
    \item \textbf{Variáveis de Interesse}: Queda de pressão ($\Delta p$, $\Delta p/L$), velocidade média, número de Reynolds, número de células, tempo de execução e resíduos de convergência
\end{itemize}

\textbf{Limitações conhecidas}:
\begin{itemize}
    \item Não aborda escoamentos compressíveis ou transientes
    \item Não inclui transferência de calor ou espécies químicas
    \item Foca em leitos cilíndricos (não trata geometrias complexas arbitrárias)
\end{itemize}

% ============================================================
% ESTRUTURA DO TRABALHO
% ============================================================
\section{Estrutura do Trabalho}

Para guiar o texto, organizo este TCC como uma narrativa do desenvolvimento desse projeto. A estrutura está organizada da seguinte forma:

\begin{itemize}
    \item \textbf{Capítulo 2 - Referencial Teórico}: Reúne o referencial teórico sobre escoamento em meios empacotados, fundamentos de CFD e práticas de verificação e validação.
    
    \item \textbf{Capítulo 3 - Materiais e Métodos}: Descreve a metodologia adotada, casos de teste e métricas de avaliação.
    
    \item \textbf{Capítulo 4 - Sistema Proposto}: Detalha a arquitetura e a implementação do pipeline: DSL, geração geométrica utilizando scripts Python, malha, automações, banco de dados, API e visualização.
    
    \item \textbf{Capítulo 5 - Resultados e Discussão}: Apresenta casos de uso, validação com a equação de Ergun, análise de performance e exemplos de estudos paramétricos.
    
    \item \textbf{Capítulo 6 - Conclusão}: Faz uma discussão sobre a validação da metodologia, traz estudos de caso, resultados, limitações encontradas e caminhos futuros, com foco em ampliar casos de uso e robustez do sistema.
\end{itemize}
