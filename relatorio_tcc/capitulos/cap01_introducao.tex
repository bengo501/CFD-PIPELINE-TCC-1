\chapter{Introdução}
\label{cap:introducao}

% ============================================================
% CONTEXTUALIZAÇÃO
% ============================================================
\section{Contextualização}

Leitos empacotados são amplamente utilizados em processos industriais químicos, petroquímicos e farmacêuticos para diversas operações unitárias, incluindo absorção, adsorção, reações catalíticas e separação de misturas. A eficiência desses processos depende fortemente das características do escoamento fluido através do meio poroso formado pelas partículas empacotadas.

A simulação computacional por Dinâmica dos Fluidos Computacional (CFD) tem se tornado uma ferramenta essencial para o projeto e otimização de leitos empacotados, permitindo a análise detalhada de campos de velocidade, pressão e temperatura sem a necessidade de experimentos físicos custosos. No entanto, o processo de configuração e execução de simulações CFD é complexo, demorado e propenso a erros, especialmente quando múltiplas configurações geométricas precisam ser avaliadas.

% ============================================================
% MOTIVAÇÃO
% ============================================================
\section{Motivação}

A configuração manual de simulações CFD de leitos empacotados envolve várias etapas repetitivas e suscetíveis a erros:

\begin{itemize}
    \item Geração de geometrias 3D com diferentes configurações de empacotamento
    \item Definição de parâmetros de simulação (propriedades do fluido, condições de contorno)
    \item Geração de malhas computacionais
    \item Configuração de solvers e esquemas numéricos
    \item Pós-processamento e extração de resultados
    \item Validação com correlações empíricas (ex: equação de Ergun)
\end{itemize}

Cada uma dessas etapas pode levar horas de trabalho manual, tornando estudos paramétricos e otimizações impraticáveis. Além disso, a reprodutibilidade dos resultados é comprometida pela falta de padronização e documentação inadequada dos processos.

% ============================================================
% PROBLEMA
% ============================================================
\section{Definição do Problema}

Atualmente, não existe uma solução integrada que automatize todo o fluxo de trabalho de simulações CFD para leitos empacotados, desde a especificação de parâmetros até a geração de resultados validados. Pesquisadores e engenheiros precisam:

\begin{enumerate}
    \item Utilizar múltiplas ferramentas de software desconexas
    \item Realizar tarefas repetitivas manualmente
    \item Gerenciar grande volume de arquivos e resultados
    \item Garantir reprodutibilidade sem ferramentas adequadas
    \item Validar resultados através de correlações empíricas
\end{enumerate}

Essa fragmentação do processo resulta em:
\begin{itemize}
    \item Baixa produtividade
    \item Erros humanos frequentes
    \item Dificuldade de colaboração entre equipes
    \item Perda de rastreabilidade de resultados
    \item Impossibilidade de realizar estudos paramétricos em larga escala
\end{itemize}

% ============================================================
% OBJETIVOS
% ============================================================
\section{Objetivos}

\subsection{Objetivo Geral}

Desenvolver um pipeline automatizado e reprodutível para simulações CFD de leitos empacotados, integrando geração de geometria, simulação numérica, armazenamento de dados e visualização de resultados em uma plataforma web unificada.

\subsection{Objetivos Específicos}

\begin{enumerate}
    \item Desenvolver uma Domain Specific Language (DSL) para descrição declarativa de parâmetros de leitos empacotados
    \item Implementar geração automática de geometrias 3D utilizando Blender com simulação física de empacotamento
    \item Integrar o OpenFOAM para automatização de simulações CFD, desde a geração de malha até a solução numérica
    \item Projetar e implementar um banco de dados relacional (PostgreSQL) para armazenamento de metadados e resultados
    \item Desenvolver um sistema de armazenamento escalável para arquivos grandes (MinIO)
    \item Criar uma API RESTful (FastAPI) para gerenciamento do pipeline
    \item Implementar um sistema de filas assíncronas para processamento de jobs em background (Redis + Celery)
    \item Desenvolver uma interface web interativa (React) para visualização e análise de resultados
    \item Containerizar toda a aplicação utilizando Docker para garantir reprodutibilidade
    \item Realizar validação dos resultados através da comparação com a equação de Ergun
\end{enumerate}

% ============================================================
% JUSTIFICATIVA
% ============================================================
\section{Justificativa}

Este trabalho se justifica pela crescente necessidade de ferramentas que aumentem a produtividade em pesquisa e desenvolvimento de processos industriais. A automação proposta oferece os seguintes benefícios:

\begin{itemize}
    \item \textbf{Reprodutibilidade}: Todos os parâmetros e configurações são documentados automaticamente
    \item \textbf{Escalabilidade}: Permite execução de dezenas ou centenas de simulações parametrizadas
    \item \textbf{Rastreabilidade}: Histórico completo de simulações e resultados em banco de dados
    \item \textbf{Colaboração}: Múltiplos usuários podem compartilhar configurações e resultados
    \item \textbf{Validação}: Comparação automática com correlações empíricas estabelecidas
    \item \textbf{Acessibilidade}: Interface web torna a ferramenta acessível sem instalação local complexa
\end{itemize}

% ============================================================
% ESTRUTURA DO TRABALHO
% ============================================================
\section{Estrutura do Trabalho}

Este trabalho está organizado da seguinte forma:

\begin{itemize}
    \item \textbf{Capítulo 2 - Referencial Teórico}: Apresenta os conceitos fundamentais sobre leitos empacotados, CFD, OpenFOAM, DSLs, arquitetura de software e tecnologias utilizadas.
    
    \item \textbf{Capítulo 3 - Materiais e Métodos}: Descreve a metodologia adotada, arquitetura do sistema, tecnologias escolhidas e justificativas das decisões de design.
    
    \item \textbf{Capítulo 4 - Desenvolvimento}: Detalha a implementação de cada componente do pipeline: DSL, geração de geometria, simulação CFD, banco de dados, API, interface web e containerização.
    
    \item \textbf{Capítulo 5 - Resultados e Discussão}: Apresenta casos de uso, validação com a equação de Ergun, análise de performance e exemplos de estudos paramétricos.
    
    \item \textbf{Capítulo 6 - Conclusão}: Sintetiza os resultados obtidos, contribuições do trabalho, limitações identificadas e perspectivas para trabalhos futuros.
\end{itemize}

