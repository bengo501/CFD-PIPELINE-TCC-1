\chapter{Cronograma}
\label{cap:cronograma}

Este capítulo apresenta o cronograma de atividades planejadas para o desenvolvimento do trabalho, dividido em duas etapas: TCC1 (semestre atual) e TCC2 (semestre seguinte).

% ============================================================
% CRONOGRAMA SIMPLIFICADO
% ============================================================
\section{Cronograma Simplificado}

A Tabela \ref{tab:cronograma_simples} apresenta a distribuição temporal das atividades principais ao longo dos 8 meses de desenvolvimento do TCC.

\begin{table}[htb]
\centering
\caption{cronograma simplificado de atividades}
\label{tab:cronograma_simples}
\resizebox{\textwidth}{!}{%
\begin{tabular}{lcccccccc}
\toprule
\textbf{atividade} & \textbf{mês 1} & \textbf{mês 2} & \textbf{mês 3} & \textbf{mês 4} & \textbf{mês 5} & \textbf{mês 6} & \textbf{mês 7} & \textbf{mês 8} \\
\midrule
A1. levantamento bibliográfico & Sem. 1–2 & & & & & & & \\
A2. especificação da DSL & Sem. 3–4 & & & & & & & \\
A3. parser/compilador da DSL & & Sem. 1–2 & & & & & & \\
A4. geração geométrica (Blender) & & Sem. 2–3 & & & & & & \\
A5. template OpenFOAM (caso base) & & Sem. 3–4 & & & & & & \\
A6. pipeline malha/solver & & & Sem. 1–2 & & & & & \\
A7. pós-processamento (variáveis) & & & Sem. 2 & & & & & \\
A8. ingestão/persistência (DB/MinIO) & & & Sem. 3 & & & & & \\
A9. API (FastAPI) & & & Sem. 3–4 & & & & & \\
A10. frontend (React/Plotly/Three.js) & & & & Sem. 1–2 & & & & \\
A11. integração ponta a ponta (E2E) & & & & Sem. 2 & & & & \\
A12. validação inicial & & & & Sem. 3 & & & & \\
A13. documentação TCC1 & & & & Sem. 3–4 & & & & \\
A14. apresentação (slides/ensaio) & & & & Sem. 4 & & & & \\
\midrule
A16. refino da DSL e compilador & & & & & Sem. 1–2 & & & \\
A17. otimizações modelagem & & & & & Sem. 2–3 & & & \\
A18. automação estudos e DOE & & & & & Sem. 3–4 & Sem. 1–4 & & \\
A19. GCI completo e validação & & & & & & Sem. 2–4 & & \\
A20. comparador no dashboard & & & & & & & Sem. 1–2 & \\
A21. relatórios automáticos & & & & & & & Sem. 2–3 & \\
A22. hardening do backend & & & & & & & Sem. 3–4 & \\
A23. performance avançada & & & & & & & Sem. 3–4 & \\
A24. pacote de replicação & & & & & & & & Sem. 1 \\
A25. monografia TCC2 & & & & & & & & Sem. 1–3 \\
A26. defesa e demonstração & & & & & & & & Sem. 3–4 \\
A27. publicação e depósito & & & & & & & & Sem. 2–4 \\
A28. auditoria reprodutibilidade & & & & & & & & Sem. 3–4 \\
\midrule
A15. gestão e checkpoints & \multicolumn{4}{c}{\textbf{semanal (TCC1)}} & & & & \\
A29. gestão e checkpoints TCC2 & & & & & \multicolumn{4}{c}{\textbf{contínuo (TCC2)}} \\
\bottomrule
\end{tabular}%
}
\end{table}

% ============================================================
% CRONOGRAMA DETALHADO
% ============================================================
\section{Cronograma Detalhado com Entregas}

A Tabela \ref{tab:cronograma_detalhado} detalha cada atividade com suas entregas principais e critérios de aceite.

\begin{landscape}
\begin{table}[htb]
\centering
\caption{cronograma detalhado de atividades, entregas e critérios de aceite}
\label{tab:cronograma_detalhado}
\resizebox{\linewidth}{!}{%
\begin{tabular}{p{6cm}p{10cm}p{8cm}}
\toprule
\textbf{atividade} & \textbf{entregas principais} & \textbf{critério de aceite} \\
\midrule
\multicolumn{3}{l}{\textbf{TCC1 - Semestre 1 (Meses 1-4)}} \\
\midrule
A1. levantamento bibliográfico & Referências (DSL, Blender API, OpenFOAM, Ergun, visualização), arquivo references.bib organizado & 10–15 referências-chave revisadas e salvas por tema \\
\midrule
A2. especificação da DSL & Definição de blocos (bed, particles, physics, export), unidades e exemplos & Rascunho da gramática + exemplos .bed válidos e inválidos com mensagens adequadas \\
\midrule
A3. parser/compilador da DSL & Implementação em Python (Lark + Pydantic); geração do params.json canônico & Mesmos .bed $\Rightarrow$ mesmo params.json; erros com linha/coluna reportados \\
\midrule
A4. geração geométrica (Blender) & Script headless: paredes, tampas, partículas com empacotamento; exportação STL & STL em escala correta e geometria manifold \\
\midrule
A5. template OpenFOAM (caso base) & Estrutura de caso base (0/, constant/, system/) com dicionários parametrizados & blockMesh e snappyHexMesh executam sem erro \\
\midrule
A6. pipeline malha/solver & Automação blockMesh $\rightarrow$ snappyHexMesh $\rightarrow$ simpleFoam, com logs e monitoramento & Solver conclui com resíduos dentro da meta no caso didático \\
\midrule
A7. pós-processamento (variáveis) & Scripts para $\Delta p$, $\Delta p/L$, velocidade média, $Re$, nº de células, tempos; globals.csv padronizado & Arquivo globals.csv com colunas fixas e valores plausíveis \\
\midrule
A8. ingestão/persistência & Metadados e variáveis no Postgres; artefatos (STL/VTK/CSV/logs) no MinIO & Execução registrada no DB com variáveis e links para arquivos \\
\midrule
A9. API (FastAPI) & Rotas: upload .bed, listar execuções, obter links de arquivos & Documentação /docs acessível e rotas testáveis via curl \\
\midrule
A10. frontend (React) & Páginas: login, novo job, lista de execuções, detalhes do job/variáveis, comparação & Usuário envia .bed, acompanha status e visualiza resultados \\
\midrule
A11. integração E2E & Pipeline completo: .bed $\rightarrow$ params.json $\rightarrow$ STL $\rightarrow$ malha/solver $\rightarrow$ variáveis $\rightarrow$ DB/API $\rightarrow$ dashboard & Caso didático roda de ponta a ponta sem intervenção manual \\
\midrule
A12. validação inicial & Estudo de malha (3 níveis) + curva $\Delta p/L$ vs. Ergun; medições de tempo de preparo & Tabela preliminar de GCI e erro $\Delta p/L$ dentro do esperado \\
\midrule
A13. documentação TCC1 & Proposta consolidada, guia rápido do usuário (2 págs), README atualizado & Revisão do orientador e entrega no prazo \\
\midrule
A14. apresentação & Slides de 10–15 min com problema, método, demo curta e resultados & Ensaio cronometrado; narrativa clara \\
\midrule
A15. gestão e checkpoints & Reuniões semanais com pautas, atas e ações registradas & Atas curtas e decisões registradas \\
\midrule
\multicolumn{3}{l}{\textbf{TCC2 - Semestre 2 (Meses 5-8)}} \\
\midrule
A16. refino DSL & Classes polidispersas, presets, validações adicionais, mensagens de erro & .bed mais expressivo; cobertura testes $\geq$ 85\% \\
\midrule
A17. otimizações Blender & Verificações automáticas manifold, formas adicionais, tempo reduzido & STL validado; redução $\sim$20\% tempo médio \\
\midrule
A18. automação DOE & Execução batelada com variações; 3 réplicas por ponto & Matriz DOE completa; reprodutibilidade documentada \\
\midrule
A19. GCI e validação & Estudo independência malha 3 níveis; comparação Ergun & Índice GCI calculado; erro $\Delta p/L$ dentro da meta \\
\midrule
A20. comparador dashboard & Telas comparação lado a lado, filtros, exportação gráficos & Usuário seleciona 2–3 execuções e exporta PDF/CSV \\
\midrule
A21. relatórios automáticos & Geração PDF templates Markdown/LaTeX; integração dashboard & Relatório reproduzível com figuras e referências \\
\midrule
A22. hardening backend & RBAC, limites taxa, logs estruturados, URLs assinadas MinIO & Testes carga e checklist segurança atendidos \\
\midrule
A23. performance avançada & Paralelização job, fila prioridades; execução HPC & Redução $\sim$25\% tempo fila em cenários carga \\
\midrule
A24. pacote replicação & Repositório scripts, dataset exemplo, docker-compose.yml & Ambiente recriado do zero por avaliador externo \\
\midrule
A25. monografia TCC2 & Texto completo resultados, discussão e conclusões, ABNT & Entrega revisada pelo orientador \\
\midrule
A26. defesa e demonstração & Slides finais, vídeo demo curta, roteiro e ensaio & Apresentação dentro do tempo e domínio conteúdo \\
\midrule
A27. publicação e depósito & Código Zenodo com DOI, licença aberta, release repositório & Artefatos acessíveis e citáveis \\
\midrule
A28. auditoria reprodutibilidade & Execução cega por colega seguindo guia replicação & Resultados dentro margem esperada \\
\midrule
A29. gestão checkpoints TCC2 & Reuniões quinzenais, atas, atualização riscos & Atas e pendências registradas \\
\bottomrule
\end{tabular}%
}
\end{table}
\end{landscape}

% ============================================================
% MARCOS E ENTREGAS
% ============================================================
\section{Marcos Principais}

\subsection{TCC1 (Semestre Atual)}

\begin{description}
    \item[Marco 1 (Fim Mês 2):] DSL compilador funcional e geração geométrica no Blender
    \item[Marco 2 (Fim Mês 3):] Pipeline CFD completo (malha + solver) e banco de dados operacional
    \item[Marco 3 (Fim Mês 4):] Sistema E2E integrado, validação inicial e documentação TCC1
\end{description}

\subsection{TCC2 (Semestre Seguinte)}

\begin{description}
    \item[Marco 4 (Fim Mês 6):] Estudos DOE completos e validação GCI/Ergun
    \item[Marco 5 (Fim Mês 7):] Dashboard avançado e hardening do sistema
    \item[Marco 6 (Fim Mês 8):] Monografia completa, defesa e publicação
\end{description}

% ============================================================
% GESTÃO DE RISCOS
% ============================================================
\section{Gestão de Riscos}

\subsection{Riscos Identificados}

\begin{enumerate}
    \item \textbf{Variabilidade do empacotamento}: Uso de física rígida pode produzir arranjos distintos
    \begin{itemize}
        \item \textit{Mitigação}: Fixar seed e executar 3 réplicas para estimar variabilidade
    \end{itemize}
    
    \item \textbf{Complexidade CFD}: Problemas de convergência ou qualidade de malha
    \begin{itemize}
        \item \textit{Mitigação}: Casos didáticos bem testados, validação incremental, GCI
    \end{itemize}
    
    \item \textbf{Adoção por usuários}: Dificuldades com detalhes de CFD e ambiente
    \begin{itemize}
        \item \textit{Mitigação}: Guia passo a passo, mensagens de erro claras, containerização
    \end{itemize}
    
    \item \textbf{Escopo muito ambicioso}: Risco de não completar todas funcionalidades
    \begin{itemize}
        \item \textit{Mitigação}: MVP funcional no TCC1, funcionalidades avançadas no TCC2
    \end{itemize}
\end{enumerate}

\subsection{Checkpoints Semanais}

Reuniões semanais com o orientador para:
\begin{itemize}
    \item Revisão de progresso
    \item Identificação de bloqueios
    \item Ajuste de escopo se necessário
    \item Validação de entregas
\end{itemize}

