\chapter{Referencial Teórico}
\label{cap:referencial}

% ============================================================
% LEITOS EMPACOTADOS
% ============================================================
\section{Leitos Empacotados}

\subsection{Definição e Aplicações}

Leitos empacotados (ou \textit{packed beds}) são estruturas compostas por partículas sólidas empilhadas em um recipiente, através das quais um fluido escoa. São amplamente utilizados em diversas operações unitárias da engenharia química e de processos \cite{ergun1952}.

\subsection{Parâmetros Característicos}

\subsubsection{Porosidade}

A porosidade $\varepsilon$ é definida como a fração de volume vazio em relação ao volume total do leito:

\begin{equation}
\varepsilon = \frac{V_{vazio}}{V_{total}}
\label{eq:porosidade}
\end{equation}

Para leitos empacotados aleatoriamente de esferas, a porosidade típica varia entre 0,36 e 0,42 \cite{dullien1992}.

\subsubsection{Perda de Carga}

A perda de carga através do leito é um parâmetro crítico de projeto. A equação de Ergun \cite{ergun1952} é amplamente utilizada para prever a queda de pressão:

\begin{equation}
\frac{\Delta P}{L} = 150 \frac{(1-\varepsilon)^2}{\varepsilon^3} \frac{\mu u_s}{d_p^2} + 1.75 \frac{(1-\varepsilon)}{\varepsilon^3} \frac{\rho u_s^2}{d_p}
\label{eq:ergun}
\end{equation}

onde:
\begin{itemize}
    \item $\Delta P$ = queda de pressão [Pa]
    \item $L$ = altura do leito [m]
    \item $\varepsilon$ = porosidade [-]
    \item $\mu$ = viscosidade dinâmica do fluido [Pa·s]
    \item $u_s$ = velocidade superficial [m/s]
    \item $d_p$ = diâmetro da partícula [m]
    \item $\rho$ = densidade do fluido [kg/m³]
\end{itemize}

% ============================================================
% CFD E OPENFOAM
% ============================================================
\section{Dinâmica dos Fluidos Computacional (CFD)}

\subsection{Fundamentos}

CFD é a ciência de predizer o escoamento de fluidos, transferência de calor e fenômenos relacionados através da solução numérica de equações governantes \cite{versteeg2007}. As equações fundamentais são:

\subsubsection{Conservação da Massa}

\begin{equation}
\frac{\partial \rho}{\partial t} + \nabla \cdot (\rho \mathbf{u}) = 0
\label{eq:continuidade}
\end{equation}

\subsubsection{Conservação da Quantidade de Movimento (Navier-Stokes)}

\begin{equation}
\frac{\partial (\rho \mathbf{u})}{\partial t} + \nabla \cdot (\rho \mathbf{u} \mathbf{u}) = -\nabla p + \nabla \cdot \boldsymbol{\tau} + \rho \mathbf{g}
\label{eq:navier-stokes}
\end{equation}

\subsection{OpenFOAM}

OpenFOAM (\textit{Open Field Operation and Manipulation}) é uma plataforma de código aberto para simulações de CFD, desenvolvida em C++. Possui as seguintes características principais:

\begin{itemize}
    \item Modular e extensível
    \item Suporte a malhas não estruturadas
    \item Diversos solvers especializados
    \item Paralelização via MPI
    \item Grande comunidade de usuários
\end{itemize}

\subsubsection{Estrutura de Casos OpenFOAM}

Um caso OpenFOAM possui a seguinte estrutura de diretórios:

\begin{lstlisting}[language=bash, caption={Estrutura de diretórios OpenFOAM}]
caso/
├── 0/              # Condições iniciais
│   ├── U           # Campo de velocidade
│   ├── p           # Campo de pressão
│   └── ...
├── constant/       # Propriedades constantes
│   ├── polyMesh/   # Malha computacional
│   ├── transportProperties
│   └── turbulenceProperties
└── system/         # Parâmetros de solução
    ├── controlDict
    ├── fvSchemes
    └── fvSolution
\end{lstlisting}

% ============================================================
% DOMAIN SPECIFIC LANGUAGES
% ============================================================
\section{Domain Specific Languages (DSL)}

\subsection{Definição}

Uma Domain Specific Language é uma linguagem de programação ou especificação dedicada a um domínio particular de aplicação \cite{fowler2010}. Diferentemente de linguagens de propósito geral (como Python ou Java), DSLs são projetadas para expressar soluções de forma concisa e legível dentro de um contexto específico.

\subsection{Vantagens}

\begin{itemize}
    \item Expressividade no domínio específico
    \item Redução de código boilerplate
    \item Validação semântica em tempo de compilação
    \item Documentação autodescritiva
    \item Menor curva de aprendizado para especialistas do domínio
\end{itemize}

\subsection{ANTLR}

ANTLR (\textit{ANother Tool for Language Recognition}) é um gerador de parsers poderoso para leitura, processamento, execução ou tradução de texto estruturado \cite{parr2013}. Permite definir gramáticas formais e gerar código para análise léxica e sintática.

% ============================================================
% ARQUITETURA DE SOFTWARE
% ============================================================
\section{Arquitetura de Software}

\subsection{Arquitetura REST}

REST (\textit{Representational State Transfer}) é um estilo arquitetural para sistemas distribuídos que utiliza o protocolo HTTP para comunicação \cite{fielding2000}. Características principais:

\begin{itemize}
    \item Stateless (sem estado)
    \item Cliente-servidor
    \item Interface uniforme
    \item Sistema em camadas
    \item Cache
\end{itemize}

\subsection{Microserviços}

Arquitetura de microserviços é um estilo que estrutura uma aplicação como uma coleção de serviços pequenos, autônomos e independentemente implantáveis \cite{newman2015}.

\subsection{Contêineres e Docker}

Contêineres são unidades de software que empacotam código e suas dependências para que a aplicação execute de forma rápida e confiável em diferentes ambientes computacionais \cite{merkel2014}.

% ============================================================
% BANCOS DE DADOS
% ============================================================
\section{Sistemas de Armazenamento}

\subsection{PostgreSQL}

PostgreSQL é um sistema de gerenciamento de banco de dados relacional objeto-relacional (ORDBMS) de código aberto \cite{stonebraker1986}. Oferece:

\begin{itemize}
    \item Conformidade com SQL padrão
    \item Transações ACID
    \item Suporte a JSON
    \item Extensibilidade
    \item Alta performance
\end{itemize}

\subsection{Object Storage (MinIO)}

MinIO é um servidor de armazenamento de objetos de alto desempenho, compatível com a API Amazon S3 \cite{minio2021}. Ideal para armazenamento de arquivos grandes e não estruturados.

% ============================================================
% FILAS E PROCESSAMENTO ASSÍNCRONO
% ============================================================
\section{Processamento Assíncrono}

\subsection{Redis}

Redis é um armazenamento de estrutura de dados em memória, usado como banco de dados, cache e message broker \cite{redis2021}. Características:

\begin{itemize}
    \item Extremamente rápido (operações em memória)
    \item Estruturas de dados versáteis
    \item Persistência opcional
    \item Pub/Sub
\end{itemize}

\subsection{Celery}

Celery é uma fila de tarefas distribuída para processar vastas quantidades de mensagens \cite{celery2021}. Permite:

\begin{itemize}
    \item Execução assíncrona de tarefas
    \item Agendamento
    \item Retry automático
    \item Monitoramento em tempo real
\end{itemize}

% ============================================================
% FRAMEWORKS WEB
% ============================================================
\section{Frameworks Web}

\subsection{FastAPI}

FastAPI é um framework web moderno e de alto desempenho para construção de APIs com Python \cite{fastapi2021}. Vantagens:

\begin{itemize}
    \item Alto desempenho (comparável a Node.js e Go)
    \item Documentação automática (OpenAPI/Swagger)
    \item Validação de dados com Pydantic
    \item Suporte assíncrono nativo
    \item Type hints completos
\end{itemize}

\subsection{React}

React é uma biblioteca JavaScript para construção de interfaces de usuário \cite{react2021}. Características:

\begin{itemize}
    \item Component-based
    \item Virtual DOM
    \item Unidirecional data flow
    \item Ecossistema rico
\end{itemize}

% ============================================================
% VISUALIZAÇÃO 3D
% ============================================================
\section{Visualização e Modelagem 3D}

\subsection{Blender}

Blender é uma suíte de criação 3D de código aberto que suporta modelagem, rigging, animação, simulação, renderização, composição e rastreamento de movimento \cite{blender2021}.

\subsubsection{Blender Python API}

A API Python do Blender (bpy) permite automação completa através de scripts, incluindo:

\begin{itemize}
    \item Criação e manipulação de geometrias
    \item Aplicação de modificadores
    \item Simulações físicas (rigid body)
    \item Exportação para diversos formatos (STL, OBJ)
\end{itemize}

\subsection{Three.js}

Three.js é uma biblioteca JavaScript para criar gráficos 3D no navegador usando WebGL \cite{threejs2021}. Permite visualização interativa de modelos 3D sem plugins.

