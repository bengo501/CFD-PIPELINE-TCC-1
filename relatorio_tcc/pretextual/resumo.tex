\begin{resumo}
As simulações de Dinâmica dos Fluidos Computacional (CFD) de leitos empacotados são amplamente utilizadas em processos industriais, mas seu processo de configuração é complexo, demorado e propenso a erros. Este trabalho apresenta o desenvolvimento de um pipeline automatizado e reprodutível que integra geração de geometria, simulação numérica, armazenamento de dados e visualização de resultados. O sistema utiliza uma Domain Specific Language (DSL) para especificação declarativa de parâmetros, processada por um compilador baseado em ANTLR. A geração de geometrias 3D é realizada automaticamente no Blender com simulação física de empacotamento, enquanto o OpenFOAM executa as simulações CFD de forma automatizada. Uma arquitetura baseada em microserviços foi implementada, utilizando FastAPI para API RESTful, PostgreSQL para metadados, MinIO para armazenamento de arquivos, Redis e Celery para processamento assíncrono, e React para interface web. A containerização com Docker garante reprodutibilidade completa. Os resultados foram validados através da comparação com a equação de Ergun, apresentando desvios inferiores a 10\%. O sistema demonstrou capacidade de executar estudos paramétricos automatizados, reduzindo o tempo de configuração de horas para minutos e garantindo rastreabilidade completa através do banco de dados. A plataforma desenvolvida representa uma contribuição significativa para a automação de simulações CFD em engenharia de processos.

\textbf{Palavras-chave}: CFD. OpenFOAM. Leitos Empacotados. Domain Specific Language. Pipeline Automatizado. Docker. Microserviços.
\end{resumo}

